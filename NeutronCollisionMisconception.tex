%\nonstopmode
\hbadness=100000
\documentclass[a4paper, 12pt]{article}
\usepackage{verbatim,amsmath,graphicx,geometry,textcomp,url,caption}
\geometry{ a4paper, total={170mm,257mm}, left=20mm, top=20mm}

\usepackage[toc, page]{appendix}
\usepackage[dvipsnames]{xcolor}
\definecolor{subr}{rgb}{0.8, 0.33, 0.0}

\begin{document}
\begin{center}
Demystifying neutron lethargy			\\
Ocean Wong								\\
2019-03-18
\end{center}

When attempting to find the solution to the following question:

"How can the average number of collisions necessary to thermalise a fission neutron (slow
it down from 2 MeV to 1 eV) in deuterium be calculated?
"

When deriving the number of collision required $N$ for a given $\alpha = (\frac{A-1}{A+1})^2$ (where $A$ is the atomic mass), several fellow students has stumbled across the following misconception:

\section{Incorrect method}\label{Incorrect}
An intutive method is to do the following:

\begin{align}
	\overline{E_f} &= {E_i} \frac{(\alpha+1)}{2} =  \int P(E) E dE \\
	\overline{\Delta u} &= ln \left( \frac{E_i}{\overline{E_f}} \right)	\\
	N &= \frac{ ln \left( \frac{2 MeV}{1eV} \right)} { \overline{\Delta u} }
\end{align}

Which corresponds to the procedure as follows:
\begin{enumerate}
	\item Find the average energy lost $\overline{E}$
	\item Calculate the lethargy change it represents
	\item Find the number of collisions (each with energy $\overline{E}$ lost) required to slow it down.
\end{enumerate}

In a single equation, this is represented as:
\begin{align}
	N=\frac{u(2MeV) - 2(1eV)}{\Delta u \left(\int E P(E) dE \right) }
	N=\frac{ln\left(2000000\right)}{ln\left(\frac{1+\alpha}{2} \right)}
\end{align}
Which gives the incorrect answer of 22.7 collisions for 2MeV neutrons moderated by Deuterium $A=2$

\section{Correct method} \label{Correct}
	However, the correct way to do so is 
\begin{align}
	\overline{\Delta u} &= \int (\Delta u) P\big(\Delta u \big) d(\Delta u)	\\
		&= \int\limits_{E_f=\alpha E_i}^{E_f=E_i} \Delta u (E_f) P\big(\Delta u (E_f) \big) \frac{d(\Delta u)}{dE_f} dE_f \label{scaleUprob}
\end{align}

where the conversion to $\Delta u$ is carried out as follows:
\begin{align}
	\Delta u (E_f) = ln\left(\frac{E_i}{E_f} \right)
\end{align}
and the probability distribution in $u$ space can be converted back into probability distribution in $E_f$ space in a straightforward manner:
\begin{align}
	P(E_f) dE_f &= P(\Delta u (E_f) ) {d(\Delta u)}	\\
	P(\Delta u (E_f) ) \frac{d(\Delta u)}{dE_f} &= P(E_f)
\end{align}
simplifying equation \ref{scaleUprob}
\begin{align}
	\overline{\Delta u} &=	\int\limits_{E_f=\alpha E_i}^{E_f=E_i} \big(\Delta u (E_f) \big) P\big(E_f \big) dE_f	\\
						&=	\int\limits_{E_f=\alpha E_i}^{E_f=E_i} ln \left( \frac{E_i}{E_f} \right) P\big(E_f \big) dE_f \\
\end{align}

Which will be evaluated to
\begin{align}
	\overline{\Delta u} &=	1+\frac{\alpha}{1+\alpha} ln(\alpha)	\\
	N &= \frac{ ln \left( \frac{2 MeV}{1eV} \right)} { \overline{\Delta u} }
\end{align}
Which gives 20.8 collisions for 2MeV neutrons moderated by Deuterium $A=2$

The numerical accuracy of this answer can be verified with the code

NeutronLethargy/MonteCarlo.py

in the repository 

https://github.com/OceanNuclear/NeutronLethargy

\section{Explanation}
	The correct method does the following:
	
	*Insert figure 1
	
	Multiplying $P(E)$ by $\Delta u(E)$ gives a quantity with the dimension of "Lethargy per unit energy".
	Integrating $\Delta u(E) P(E)$ gives the mean lethargy change.

	
	Alternatively, if one wishes to work with $\Delta u$,
	
	*Insert Figure 2
	
	In figure \ref{Fig2}, multiplying $P(u)$ 
	Integrating $\Delta u(E) P(E)$ gives the mean lethargy change.

	
	However, in section \ref{Incorrect}, the following method is used (See figure \ref{Fig3}) :
	*Insert Figure 3
	Due to the non-linearity of the $\Delta u$ vs $E$ a small 'excursion' to the left will contribute a much larger increment in lethargy gain than a small 'excursion' to the right.
	
	Note that if $\alpha$ is very close to 1, then this non-linearity can be ignored, i.e. at very large atomic masses, e.g. ${}^{238}$U, the above two methods should give very similar results.

	This means that a collision losing slightly more than $\overline{E}$ requires multiple collision losing slightly less energy than $\overline{E}$ to counter its effect.
	This problem is actually a confusion between arithematic mean (section \ref{Incorrect}) and geometric mean (section \ref{Correct}) in disguise. The former method falsely assumes that an arithmatic mean can be done, finding the arithmatic average of the energy lost, instead of the geometric average of $E$ (which is identical to the arithmetic average of $\Delta u$).

\section{Acknowledgement}
	Many thanks to Sam Dyson for his helpful input in discussing and assisting the formulation of this paper.

\end{document}