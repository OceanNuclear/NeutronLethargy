%\nonstopmode
\hbadness=100000
\documentclass[a4paper, 12pt]{article}
\usepackage{verbatim,amsmath,graphicx,geometry,textcomp,url,caption}
\geometry{ a4paper, total={170mm,257mm}, left=20mm, top=20mm}

\usepackage[toc, page]{appendix}
\usepackage[dvipsnames]{xcolor}
\definecolor{subr}{rgb}{0.8, 0.33, 0.0}

\begin{document}
\begin{center}
Demystifying neutron lethargy			\\
Ocean Wong								\\
2019-03-18
\end{center}

When attempting to find the solution to the following question:
"
How can the average number of collisions necessary to thermalise a fission neutron (slow
it down from 2 MeV to 1 eV) in deuterium be calculated?
"
Several fellow students has stubled across the following misconception.

\section{Incorrect method}
An intutive method is to do the following:

\begin{align}
	\overline{E_f} &= {E_i} \frac{(\alpha+1)}{2} =  \int P(E) E dE \\
	\overline{\Delta u} &= ln \left( \frac{E_i}{\overline{E_f}} \right)	\\
	N &= \frac{\left(\Delta u () \right)} {\left( \overline{\Delta u} \right)}
\end{align}

Which corresponds to the procedure as follows:
\begin{enumerate}
	\item Find the average energy lost $\bar{E}$
	\item Calculate the lethargy change it represents
	\item Find the number of collisions (each with energy $\bar{E}$ lost) required to slow it down.
\end{enumerate}

In a single equation, this is represented as:
\begin{align}
	N=\frac{\Delta u ( \frac{}{} )}{\Delta u \left(\int P(E) E dE \right) }
\end{align}

\section{Correct method}
	However, this isn't correct. We know that the size of the

\section{Explanation}

	One small excursion to the left will contribute a much larger increment in lethargy gain than a small excursion to the right. 
	This means that a collision losing slightly more than $\bar{E}$ requires multiple collision losing slightly less energy than $\bar{E}$ to counter its effect.

\subsection{Arithmatic mean Geometric mean}
	Why geometric mean should be applied

\section{Acknowledgement}
	Thank you for Sam Dyson from 2018-19 class of PTNR

\end{document}